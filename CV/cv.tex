\documentclass[a4paper,10pt]{article}
\usepackage{hyperref,fancyhdr,enumitem,color}
\usepackage[a4paper,text={7in,10in}]{geometry}
\usepackage{pslatex}
\usepackage{longtable}


\begin{document}

\setlength\LTleft{0.2in}
\setlength\LTright{1in plus \fill}

\begin{center}
\textbf{\large{Frank Y. Wang}} 
\end{center}

\begin{tabular*}{0.95 \textwidth} { l l @{\extracolsep{\fill}} r}
frankw@mit.edu & & http://frankwang.org\\
(408) 893-1709 & & \\ \\
\textbf{Education} & Massachusetts Institute of Technology & Sept 2012 - \\ 
 & Ph.D. in Computer Science & \\
 & Advised by Prof. Nickolai Zeldovich and Prof. James Mickens & \\
 \\
 & Stanford University & Sept 2008 - June 2012 \\
 & B.S. in Computer Science with Honors and Minor in Mathematics & \\
 & Advised by Prof. Dan Boneh & \\ 
 & Thesis Title: Offloading Critical Security Operations to the GPU &\\ \\
\end{tabular*}

\begin{longtable}{ p{0.9in} l }
\textbf{Interests} & Computer Security and Privacy, Systems, and Applied Cryptography \\ \\
\textbf{Research Projects} & \begin{minipage}[t]{0.78 \columnwidth}
\textbf{Cryptographically Enforced Access Control for User Data in
Untrusted Clouds}. \\ Sieve is a new platform which
selectively (and securely) exposes user data to web services.
Sieve has a user-centric storage model: each user
uploads encrypted data to a single cloud store, and by
default, only the user knows the decryption keys. Given
this storage model, Sieve defines an infrastructure to support
rich, legacy web applications. Using attribute-based
encryption, Sieve allows users to define intuitively understandable
access policies that are cryptographically
enforceable. Using key homomorphism, Sieve can re-encrypt
user data on storage providers in situ, revoking
decryption keys from web services without revealing new
keys to the storage provider. Using secret sharing and
two-factor authentication, Sieve protects cryptographic
secrets against the loss of user devices like smartphones
and laptops. The result is that users can enjoy rich, legacy
web applications, while benefiting from cryptographically
strong controls over which data a web service can access. \\
\end{minipage} \tabularnewline

\textbf{Publications} & \begin{minipage}[t]{0.78 \textwidth} 
\begin{enumerate}[leftmargin=*]
\setlength{\itemsep}{7pt}
                \setlength{\parskip}{0pt}
                \setlength{\parsep}{0pt}

\item{\textbf{Frank Wang}, James Mickens, Nickolai Zeldovich, and  Vinod Vaikuntanathan. ``Sieve: Cryptographically Enforced Access Control for User Data in Untrusted Clouds." In proceedings of Networked Systems Design and Implementation (NSDI) 2016.}
               
\item{Dan Boneh, Craig Gentry, Shai Halevi, \textbf{Frank Wang}, and David J. Wu. ``Private database queries using Somewhat Homomorphic Encryption." In proceedings of Applied Cryptography and Network Security (ACNS) 2013.}

\item{Zachary Weinberg, Jeffrey Wang, Vinod Yegneswaran, Linda Briesemeister, Steven Cheung, \textbf{Frank Wang}, and Dan Boneh. ``StegoTorus: a camouflage proxy for the Tor anonymity system." In proceedings of ACM Conference on Computer and Communications Security (CCS) 2012.}

\item{Peifung E. Lam, John C. Mitchell, Andre Scedrov, Sharada Sundaram, and \textbf{Frank Wang}. ``Declarative privacy policy: Finite models and Attribute-Based Encryption." In proceedings of 2nd ACM International Health Informatics Symposium, 2012.}

\item{Venkatasubramanian Viswanthan and \textbf{Frank Wang}. ``Theoretical analysis of the effect of particle size and support on the kinetics of oxygen reduction reaction on platinum nanoparticles." Nanoscale, 2012, 4(16), 5110-5117.}

\item{Venkatasubramanian Viswanathan, \textbf{Frank Wang}, and Heinz Pitsch. ``Dynamic Monte-Carlo based approach for simulating nanostructured catalytic and electrocatalytic systems." Computing in Science and Engineering, 2012, 14(2), 60-68.}

\end{enumerate}
\end{minipage} \tabularnewline

\end{longtable} 

\begin{tabular*}{0.95 \textwidth} { p{0.9in} l @{\extracolsep{\fill}} r}

\textbf{Invited Talks} & Boston University Security Seminar & Oct 2016 \\
& \begin{minipage}[t]{0.6 \textwidth}
\textit{Sieve: Cryptographically Enforced Access Control for User Data in Untrusted Clouds} \end{minipage} & \\ \\
& Georgia Institute of Technology Cybersecurity Lecture Series & Sept 2016 \\
& \begin{minipage}[t]{0.6 \textwidth}
\textit{Sieve: Cryptographically Enforced Access Control for User Data in Untrusted Clouds} \end{minipage} & \\ \\
& University of Washington Computer Science Seminar& Aug 2016 \\
& \begin{minipage}[t]{0.6 \textwidth}
\textit{Sieve: Cryptographically Enforced Access Control for User Data in Untrusted Clouds} \end{minipage} & \\ \\
& Curry-On 2016 & July 2016 \\
& \begin{minipage}[t]{0.6 \textwidth} 
\textit{Sieve: Cryptographically Enforced Access Control for User Data in Untrusted Clouds} \end{minipage} & \\ \\
& New England Security Day & Apr 2016 \\
& \begin{minipage}[t]{0.6 \textwidth}
\textit{Sieve: Cryptographically Enforced Access Control for User Data in Untrusted Clouds} \end{minipage} & \\ \\
& 2016 Networked Systems Design and Implementation (NSDI) & Mar 2016 \\
& \begin{minipage}[t]{0.6 \textwidth} 
\textit{Sieve: Cryptographically Enforced Access Control for User Data in Untrusted Clouds} \end{minipage} & \\ \\
& Northeastern Security Seminar & July 2015 \\
& \begin{minipage}[t]{0.6 \textwidth} \textit{Sieve: Cryptographically Enforced Access Control for User Data in Untrusted Clouds} \end{minipage} & \\ \\

\textbf{External} & 2016 IEEE S\&P, 2016 ACM CCS & \\
\textbf{Reviewer} & & \\ \\

\textbf{Teaching} 
& Teaching Assistant for MIT Computer Networks (6.829) & Fall 2016 \\
& Teaching Assistant for MIT Computer Systems Security (6.858) & Fall 2012 \\
& Teaching Assistant for Stanford Introduction to Cryptography (CS 255) & Winter 2012 \\ \\

\textbf{Other} & \textbf{Recurse Center}, Resident & May 2016, June 2015 \\ 
\textbf{Teaching} & \begin{minipage}[t]{0.5 \textwidth} 
Held security and cryptography seminars. Mentored programmers 
in security, cryptography, and systems. 
\end{minipage} \\ \\

 & \textbf{CMU Coursera Class}, Online Course Assistant & 2014 - Current\\
 & Statistical Thermodynamics: Molecules to Machines & \\ \\

\textbf{Awards} & \textbf{NSF Graduate Research Fellowship} & Sept 2013 - Sept 2016\\
& \textbf{Jacobs Presidential Fellowship} & Sept 2012 - June 2013\\
& Massachusetts Institute of Technology \\
\\

\textbf{Industry}  & \textbf{Facebook}, Security Engineer Research Intern & June - Aug 2013 \\
\textbf{Experience} & \begin{minipage}[t]{0.5 \textwidth} 
Designing and building infrastructure for intrusion detection.
\end{minipage}
& \\ \\
& \textbf{Google}, Software Engineering Intern & June - Aug 2012 \\
& \begin{minipage}[t]{0.5 \textwidth} 
Chrome security research project for Security Research Team under mentorship of Elie Bursztein and Ulfar Erlingsson.
\end{minipage}
& \\ \\

\textbf{Other} & \textbf{Rough Draft Ventures}, Student Partner & Sept 2014 - Current \\
\textbf{Experience} & \begin{minipage}[t]{0.5 \textwidth}
  Investment fund that invests in Boston early student startups.
  \end{minipage} & \\ \\
& \textbf{Sidney Pacific}, Board of Trustees & May 2015 - Current \\
  & \begin{minipage}[t]{0.5 \textwidth}
  Advise new officers, administer elections, and deal with major issues.
  \end{minipage} & \\ \\
  & \textbf{Cybersecurity Factory}, Co-founder & 2015 - Current \\
  & \begin{minipage}[t]{0.5 \textwidth}
  Summer program that provides mentorship and capital to early stage
  cybersecurity companies.
  \end{minipage} & \\ \\
  & \textbf{Sidney Pacific}, Web Chair & Sept 2012 - May 2015 \\
 & \begin{minipage}[t]{0.5 \textwidth}
Maintain website and create new features.
\end{minipage}
& \\ \\

\end{tabular*}

\end{document}
